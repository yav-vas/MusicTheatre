\documentclass[12pt]{article}
\usepackage[utf8]{inputenc}
\usepackage[T2A]{fontenc}
\usepackage[bulgarian]{babel}
\usepackage{geometry}
\geometry{
    a4paper,
    left=20mm,
    right=20mm,
    top=20mm,
    bottom=20mm,
}

\title{Документация на проекта Music Theatre}
\author{Явор Василев}
\date{22 януари 2023 г.}

\begin{document}

\maketitle

\section{Цел и основни функционалности}

    Проектът има за цел да моделира система за управление на музикален театър. Чрез основните принципи на ООП и с помощта на езика Java се създава конзолно приложение, което обработва информацията въведена от потребителя и изпълнява зададените команди.
    Ключовите функционалности включват:

    \begin{enumerate}
        \item Създаване на зали и постановки
        \item Отмяна на постановки
        \item Продажба на билети на каса
        \item Закупуване на билети от потребители
        \item Отказване на билети
        \item Проверяване на закупените билети
    \end{enumerate}

    Предвид сложността на имплементацията на всички тези функционалности се наложи използването на множество класове, връзките между които ще бъдат обяснени по-долу.

\section{Структура на проекта}

    При стартиране на програмата автоматично се извиква методът за регистрация на администратор. След успешна регистратиця този администратор автоматично се логва в системата и се извиква метод за посрещането му. Този метод за посрещане му дава възможност да избере едно от действията, които са му позволени да извърши. При излизане от профила системата се връща в начално състояние. От това състояние може или да се регистрира нов потребител, или да се влезе в профил. При регистрация автоматично се влиза в съответния профил. Така отново се извиква методът за посрещане на потребителя и се повтаря описаното по-горе. При избиране на опция за излизане от програмата от началното меню, то цялата информация се изтрива и програмата приключва.

    Има четири вида потребители със съответните функционалности:

    \begin{enumerate}
        \item Администратор:
            \begin{enumerate}
                \item Създаване на зала
                \item Създаване на постановка
                \item Отмяна на постановка
            \end{enumerate}
        \item Продавач на билети
            \begin{enumerate}
                \item Продаване на билет
                \item Отказ на билет
            \end{enumerate}
        \item Потребител, купуващ билети
            \begin{enumerate}
                \item Закупуване на билет
                \item Отказ на билет
                \item Показване на всички закупени билети
            \end{enumerate}
        \item Проверяващ билетите
            \begin{enumerate}
                \item Проверка на билет по негов номер
            \end{enumerate}
    \end{enumerate}

\section{Структура на класовете}

    В основата на проекта стоят два абстрактни класа User и Ticket, класът Performance, класът Hall и класът Seat. При създаване на потребител се създава обект от тип User, използвайки специфичния за вида потребител конструктор от наследиците на класа User: Admin, Cashier, Customer, Checker. Всеки обект от тип Performance има своя зала (обект на класа Hall), масив от места (обекти на класа Seat) и други не дотолкова важни за общата картина характеристики. Всяко място от своя страна има билет (обект на класа Ticket), зала и постановка. Всеки билет има постановка и място. Билетите се делят на два вида - OnlineTicket и PaperTicket. Онлайн билет се създава от Customer, а хартиен - от Cashier

    Данните в приложението се съхраняват в масиви с променлива дължина (ArrayList), като има четири такива масива:

    \begin{enumerate}
        \item ArrayList<User> users
        \item ArrayList<Hall> halls
        \item ArrayList<Performance> performances
        \item ArrayList<Ticket> tickets
    \end{enumerate}

    Посредством тези масиви изключително удобно може да се провери дали съществува напр. зала със същото име, билет с даден номер и т.н. Също така, тъй като обектите на всички класове са референтни типове данни, то всички промени по обектите, ще се отразяват и в тези масиви, защото обекти на класовете извън този масив няма да съществуват. Това е гарантирано, тъй като във всички конструктори създаденият обект се добавя в масива. Т.е. всички обекти на класовете имат референция в тези масиви.

\section{Класът User}

    Класът User е абстрактен клас, който скицира функционалностите на всички потребители. Имплементира интерфейса Unique.

    \subsection{Xарактеристики}

        \begin{enumerate}
            \item Статични константи за изисквания за потребителското име и паролата
            \item username - обект от тип String
            \item password - обект от тип String
            \item registrationDate - обект от тип Date, стойност равна на датата в момента на създаване на потребителя
        \end{enumerate}

    \subsection{Конструктори}

        \begin{enumerate}
            \item User() - създава празен обект
            \item User(String, String) - дава стойности на username, password и registrationDate; Проверява дали потребителят е уникален т.е. дали не фигурира в масива users. Това се дължи на имплементацията на интерфейса Unique.
        \end{enumerate}

    \subsubsection{Медоти}

        \begin{enumerate}
            \item Get методи за полетата; getPassword() е private от съображения за сигурност
            \item void setUsername(String) - дава стойност на username, хвърля изключение ако не са изпълнени изискванията
            \item void setPassword(String) - дава стойност на password, хвърла изключение ако не са изпълнени изискванията
            \item static User registerUser(Scanner) - показва меню за избор на тип на потребител за регистрация и извиква метода за регистрация на наследника, който презаписва този метод
            \item abstract void welcomeUser() - абстрактен метод за посрещане на потребителя и показване на всички възможни действия
            \item static String inputPassword() - метод за въвеждане на парола във външен прозорец
            \item static User loginUser(Scanner) - метод за логване на потребител; изисква потребителско име и парола; връща обект от тип User
            \item booolean isUnique(User) - проверява дали обектът, подаден като параметър се съдържа в масива users; при намерено потребителско име хвърля изключение
        \end{enumerate}

\section{Класът Admin}

    Класът Admin е наследник на абстрактния клас User и позволява регистрация на администратор и изпълняване на съответните му функционалности.

    \subsection{Xарактеристики}

        \begin{enumerate}
            \item Друга стойност на статичната константа за минимална дължина на паролата
        \end{enumerate}

    \subsection{Конструктори}

        \begin{enumerate}
            \item Admin(String, String) - извиква конструктора на суперкласа със същите параметри
        \end{enumerate}

    \subsection{Методи}

        \begin{enumerate}
            \item void setPassword(String) - проверява дали е спазено допълнителното условие за паролата (дължина по-голяма от константата) и извиква същия метод на суперкласа
            \item static User registerUser(Scanner) - въвежда се потребителско име и парола и се връща обект от тип User, с действителен тип Admin
            \item void welcomeUser(Scanner) - въвежда се опция от менюто и се вика съответния метод
            \item  void createHall(Scanner) - създава зала за представления, като се задават въпроси за име на залата, брой редове и брой места на всеки ред; към момента може да се създават само правоъгълни зали
            \item  void createPerformance(Scanner) - създава постановка, като се избира зала, заглавие на постановката и други детайли; Тук се избира и каква да е цената на всяко място в залата, като може да се избира отделна категория за всеки ред. Типовете места са в enum SeatClass, като са фиксирани на три броя. Избира се и цена за всеки клас място.
            \item  void cancelPerformance(Scanner) - отменя постановка, като всеки билет за тази постановка се маркира като невалиден
        \end{enumerate}

\section{Kласът Cashier}

    Класът Cashier е наследник на абстрактния клас User и позволява регистрация на продавач на билети и изпълняване на съответните му функционалности.

    \subsection{Xарактеристики}

        \begin{enumerate}
            \item Няма специфични характеристики
        \end{enumerate}

    \subsection{Конструктори}

        \begin{enumerate}
            \item Cashier(String, String) - извиква конструктора на суперкласа със същите параметри
        \end{enumerate}

    \subsection{Методи}

        \begin{enumerate}
            \item static User registerUser(Scanner) - въвежда се потребителско име и парола и се връща обект от тип User, с действителен тип Cashier
            \item void welcomeUser(Scanner) - въвежда се опция от менюто и се вика съответния метод
            \item  void sellTicket(Scanner) - избира се постановка и място и се създава обект от деклариран тип Ticket, но действителен тип - PaperTicket
            \item  void cancelTicket(Scanner) - въвежда се номер на билет и се извиква метод за отказване на билета; възстановява се част от сумата на билета
        \end{enumerate}

\section{Класът Customer}

    Класът Customer е наследник на абстрактния клас User и позволява регистрация на клиент и изпълняване на съответните му функционалности.

    \subsection{Характеристики}

        \begin{enumerate}
            \item Статична константа за максимална дължина на имейл
            \item String email - обект от тип String; при отмяна на постановка, за която има закупен билет или при отказване на билет се изпраща писмо на този имейл (посредством съобщение в конзолата - реален имейл не се праща)
            \item String realName - обект от тип String; реалното име се отбелязва на закупения билет (обект от тип OnlineTicket)
            \item ArrayList<Ticket> ticketsOfUser - използва се за съхранение на всички билети на съответния потребител и тяхното печатане; съхранява същата референция като тази, използвана в масива в Ticket класа
        \end{enumerate}

    \subsection{Конструктори}

        \begin{enumerate}
            \item Customer(String, String, String, String) - създава обект от тип Customer, първо извиква конструктора на суперкласа и след това задава стойности на email и realName; инициализира празен ArrayList от билети за потребителя
        \end{enumerate}

    \subsection{Методи}

        \begin{enumerate}
            \item void setEmail(String) - дължина по-малка от тази на константата иначе хвърля изключение
            \item void setRealName(String) - само букви и интервали иначе хвърля изключение
            \item static User registerUser(Scanner) - въвежда се потребителско име и парола и се връща обект от тип User, с действителен тип Customer
            \item void welcomeUser(Scanner) - въвежда се опция от менюто и се вика съответния метод
            \item void buyTicket(Scanner) - закупуване на OnlineTicket; добавяне в ticketsOfUser
            \item void cancelTicket(Scanner) - въвежда се номер на билет и се извиква метод за отказване на билета
            \item void showTickets() - извежда всички валидни билети, закупени от този потребител
        \end{enumerate}

\section{Класът Checker}

    Класът Checker е наследник на абстрактния клас User и позволява регистрация на лице, проверяващо билетите и изпълняване на съответните му функционалности.

    \subsection{Характеристики}

        \begin{enumerate}
            \item Статична константа за максимална дължина на имейл
            \item String realName - обект от тип String; реалното име се отбелязва на билета при негова проверка
            \item ArrayList<Ticket> ticketsOfUser - използва се за съхранение на всички билети на съответния потребител и тяхното печатане; съхранява същата референция като тази, използвана в масива в Ticket класа
        \end{enumerate}

    \subsection{Конструктори}

        \begin{enumerate}
            \item Customer(String, String, String, String) - създава обект от тип Customer, първо извиква конструктора на суперкласа и след това задава стойности на email и realName; инициализира празен ArrayList от билети за потребителя
        \end{enumerate}

    \subsection{Методи}

        \begin{enumerate}
            \item void setEmail(String) - дължина по-малка от тази на константата иначе хвърля изключение
            \item void setRealName(String) - само букви и интервали иначе хвърля изключение
            \item static User registerUser(Scanner) - въвежда се потребителско име и парола и се връща обект от тип User, с действителен тип Customer
            \item void welcomeUser(Scanner) - въвежда се опция от менюто и се вика съответния метод
            \item void buyTicket(Scanner) - закупуване на OnlineTicket; добавяне в ticketsOfUser
            \item void cancelTicket(Scanner) - въвежда се номер на билет и се извиква метод за отказване на билета
            \item void showTickets() - извежда всички валидни билети, закупени от този потребител
        \end{enumerate}


\end{document}
